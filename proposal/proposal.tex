\documentclass[oneside,twocolumn,7pt,a4paper]{article}

\title{Efficiency of OpenFlow in Carrier Ethernet networks \\[0.3cm]
\large{Research Proposal}}
\author{Michiel Appelman}

\usepackage{url}
\usepackage{fullpage}
\usepackage{paralist}
%\usepackage{natbib}
\usepackage{color,graphicx}
\usepackage[colorlinks,linkcolor=black,citecolor=black,urlcolor=black]{hyperref}

\begin{document}
	\maketitle
	
	\section{Introduction} % (fold)
	\label{sec:introduction}
	The current state of large scale networks is defined by a stack of networking protocols which were extensively amended to over the course of years to get control over the increasing scales and fix issues with previous protocols. This has resulted in rigid network architectures which can not be innovated upon without drastic changes and high risks of breaking existing services \cite{diversifiedinternet,overcomingimpasse}. A solution to this impasse has been proposed in the form of Software Defined Networking (SDN) and network virtualization. This fundamentally different way of creating and managing network architectures yields two highly desirable features to potentially solve the ossification. 

	SDN, or `network programmability,' allows for network administrators to get a programmable interface to their network devices. This would make it possible for them to develop and deploy new and custom protocols and architectures on their networks without having to rely on vendors. Thereby again being able to innovate on their own network. Architectural challenges like getting rid of middleboxes or adding Operations, Administration and Management (OAM) features might thus be solved by operators themselves. 
	Network programmability has been implemented by the OpenFlow specification \cite{openflow}, which provides an open Application Programming Interface (API) towards the forwarding plane of networking devices. By matching on certain header fields in the frames different actions can be performed on each frame passing through the device. The actions are governed by the applications running on an OpenFlow controller (an external piece of software analogous to the control plane in a traditional environment) and can range from contemporary implementations to future protocols not yet implemented by the network equipment vendors.

	Network virtualization has been the topic of research since the idea of a diversified Internet was first talked about \cite{diversifiedinternet}. In the mean time a multitude of different implementations have been developed and are running on some sort of testbed. None of these have however led to adoption in the Internet backbone mostly because the Internet backbone already has some solutions for network virtualization and separation (\textsl{eg.} L2 and L3VPNs) \cite{net-virt10}. 
	
	Although OpenFlow does not promise network virtualization in itself, it can still be implemented using the protocol. This has already been done through FlowVisor \cite{flowvisor} and in a more scalable way using FlowN \cite{drutskoy2012scalable} but it is unclear how these implementations would perform and hold up in a production environment dealing with a multitude of VPNs. These networks are labelled as Carrier Ethernet Service Providers and can provide VPN or pseudo-wire services on layer-2 and 3 of the OSI model. They are accountable for a huge amount of data flowing over the Internet and due to their strict network flow separation, would be a perfect user of network virtualization.
			
		\subsection*{Research Question} % (fold)
		\label{sub:research_question}
		Network programmability and thus OpenFlow promises a completely open and flexible implementation of any networking architecture, including the contemporary technologies used in the aforementioned Carrier Ethernet Service Providers. It is however uncertain if and how OpenFlow will provide the same services and with what relative performance. Also it remains to be seen if SDN will indeed replace the ossified network architecture or that it will be just another technique in the toolset of the network engineer.
		
		 This research will focus on clearing up these uncertainties and give answer to the question: \textsl{``How will OpenFlow be able to provide network virtualization and thereby relieving the current network architectures of their ossified status quo?''}		

		% subsection research_question (end)
		
		\subsection*{Scope} % (fold)
		\label{sub:scope}
		Initially this research will focus on a single use-case, namely the implementation of multi-domain VPNs on the Internet. A possible implementation of this would touch on a wide variety of SDN and network virtualization techniques and would thus give a broad fairly complete image of the possibilities of this architecture.
		
		Should there be any time left, other use-cases will be composed together with the supervisor to fill in this void.
		% subsection scope (end)
		
		\subsection*{Approach} % (fold)
		\label{sub:approach}
		This research will focus on the use case of multi-domain VPNs in the Internet backbone. First we will focus on how these can be implemented using contemporary protocols, then how this can translated to a SDN specific environment and finally how much these two implementations differ form each other with regards to flexibility and manageability.
		% subsection approach (end)

	% section introduction (end)

	
	\section{Planning} % (fold)
	\label{sec:planning}
		This project will be carried out over a period of four weeks at the SNE Research Group at the University of Amsterdam. A rough timeline of the research is given in the table below: \\
		
		\begin{tabular}{l p{5.8cm}}
			Week & Task \\ \hline
				1 & Write proposal;\\
					& Formulate research questions; \\
					& Get approval for proposal; \\ \hline
				2 \& 3 & Start research; \\ 
					& Find documentation of other research/experiences within community; \\
					& Work out use-case, hypotheses and experiment; \\
				 	& Produce proof of concept; \\  \hline
				4 & Write report; \\ 
					& Prepare presentation; \\
					& Filter report to a scientific paper; \\
		\end{tabular}
		
		\subsection*{Products} % (fold)
		\label{sub:products}
			The research will result in the following products: 
			\begin{inparaenum}[\itshape a\upshape)]
				\item a report written for the System and Network Engineering master education describing the research, the process and results;
				%\item a paper written for the scientific community to be published together with other scientist in the SNE research group; and
				\item a 20 minute presentation about the research to be held on July 3rd at the University of Amsterdam.
			\end{inparaenum}
			Also, if preliminary results and available time allow for it, a proof of concept will be produced to provide the community with a sample implementation to test and adapt.
		% subsection products (end)
	% section planning (end)
	
	\bibliographystyle{ieeetr}
	\bibliography{./bibliography}
	
\end{document}