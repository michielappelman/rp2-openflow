% (c) 2014 Michiel Appelman
\section{Results} % (fold)
\label{sec:results}

The features and limitations of the three discussed technologies in Section~\ref{sec:implementation} are given in Table~\ref{tb:reqs}. In this section we will compare the contemporary technologies with the OpenFlow implementation when implementing the \ac{dvpn} service as discribed in Section~\ref{sec:dvpns}.

\begin{table}[h]
	\centering
	\begin{tabular}{r|lll}
	 & \acs{spb} & \acs{mpls} & OpenFlow / \acs{sdn}\\
	\hline
	Tagging of VPN Traffic & \acs{pbb} & \acs{vpls} (\acs{mpls}) & \acs{pbb} / \acs{mpls}\\
	MAC Scalability & yes & yes & yes\\
	Topology Discovery & \acs{isis} & \acs{ospf} & application\\
	Path Provisioning & \acs{spt} & \acs{rsvp} / \acs{ldp} & application\\
	Traffic Engineering & limited & \acs{rsvp} & application\\
	\ac{ecmp} & limited & yes & yes, using Groups\\
	\ac{bum} limiting & dependent on \acs{hw} & dependent on \acs{hw} & yes, using Metering\\
	Exchange \acsp{cmac} & no & \ac{evpn} (draft) & application\\
	Ingress Rate Limiting & dependent on \acs{hw} & dependent on \acs{hw} & yes, using Queues or Metering\\
	Fast Failover & no & \acs{frr} & yes, using Groups\\
	\acs{oam} & 802.1ag / Y.1731 & \acs{lsp} Ping / \acs{bfd} & application\\
	\hline
	Forwarding Decision & \acs{pbb} tags & \acs{mpls} labels & flow entry \\
	\ac{bum} traffic handling & flood & flood & sent to controller\\
	\end{tabular}
	\caption{Feature requirements available in discussed technologies.}
	\label{tb:reqs}
\end{table}


\subsection{\acs{spb}} % (fold)
\label{sub:r-spb}

The \ac{spb} architecture allows for a scalable carrier network supporting thousands, even millions of \acp{dvpn} using the \ac{isid} in the \ac{pbb} frame. From our theoretical implementation in Section~\ref{ssub:spb}, it became evident that setting up a \acp{dvpn} requires little configuration. The \ac{isid} only needs to be configured on the \ac{pe} connecting the \ac{ce} port and the \ac{isis} routing protocol distributes this binding to the other corresponding \acp{pe}. Another benefit is the use of Ethernet \ac{oam} standards that are mature and extensive to allow for precise monitoring and troubleshooting.

However, the simplicity of the architecture comes at a cost. The protocols has:
\begin{itemize}
	\item limited explicit or constraints-based routing, meaning few \ac{te} features,
	\item limited \ac{ecmp} functionality due to the infancy of the standard to support it, and
	\item because, failure recovery depends on \ac{isis} reconvergence, no fast failover.
\end{itemize}

These limitations are being worked on by the community, e.g.\ IEEE 802.1Qbp which provides extensive \ac{ecmp} functions. And since the technology has only been officially standardized since March 2012, it will also need to mature before it is suitable for carrier implementations.  

Because of the shortcomings of \ac{spb} with regards to the use-case set forth in Section~\ref{sec:dvpns}, we will omit this technology in our comparison. Instead we will focus on comparing the \ac{mpls} setup from Section~\ref{ssub:mpls} with the \acs{sdn}/OpenFlow architecture as designed in Section~\ref{sub:openflow}.

% subsection r-spb (end)

\subsection{Comparison} % (fold)
\label{sub:comparison}

To get an overview of the key differences between the two architectures we will follow the structure of the \ac{dvpn} requirement list defined in Section~\ref{sec:dvpns}. 

\subsubsection{Service} % (fold)
\label{ssub:service}

From a customer point-of-view it should be of no concern how the \ac{dvpn} service is implemented on the provider network. Moreover, the \acp{pe} and the rest of provider network should be completely transparent. As such, the two technologies do not show any difference in their implementation. Both technologies are able to 
\begin{inparaenum}[\itshape 1\upshape)]
	\item provide a Layer 2 broadcast domain,
	\item connect \acp{ce} without any required \ac{vpn} configuration on them, and
	\item be completely transparent to the \acp{ce}.
\end{inparaenum}

% subsubsection service (end)

\subsubsection{Transport} % (fold)
\label{ssub:transport}



MPLS:
\begin{itemize}
	\item large stack of labels (MPLS, VPLS, FRR)
	\item MAC learning in control in draft E-VPN
\end{itemize}

OF:
\begin{itemize}
	\item Lacks standard for liveness monitoring in forwarding plane - and over complete paths.
	\item MAC Scalability only from 1.0 onwards.
	\item 1.0 supported almost everywhere, 1.1 and 1.2 are not. 1.3 slowly coming.
\end{itemize}

% subsubsection transport (end)

\subsubsection{Provisioning} % (fold)
\label{ssub:provisioning}

MPLS:
\begin{itemize}
	\item initial setup complicated
	\item DVPN setup: only each PE with member port
	\item 
\end{itemize}

OF:
\begin{itemize}
	\item initial setup nonexistent, no VPNs = no flows (except LLDP/OAM)
	\item DVPN setup: every PE with member port + Ps in path
\end{itemize}

\HRule

The strength of these applications however, is the fact that operators can integrate their \ac{nms} and control plane even more. This allows for a more granular control over their traffic.

MPLS automation software written for DVPNs, but not due to lack of programmable consistent interface to HW, not portable to other vendor, sometimes even model!

complexity high due to intricate dependencies of different protocols

\HRule

OF applications need to solve from ground up, topology, etc... nortbound interface undefined, limited portability of apps between controllers. 


\ac{mpls} \acp{vpn} in OpenFlow: \cite{mpls-vpn-openflow}

\ac{mpls} control plane in OpenFlow: \cite{mpls-open}

Also: access layer intelligence.


% subsubsection provisioning (end)





% subsection comparison (end)





% section results (end)