% (c) 2014 Michiel Appelman
\section{Results} % (fold)
\label{sec:results}

\subsection{Contemporary Implementations} % (fold)
\label{sub:contemporary_implementations}

%%% ATM %%
%
%As mentioned, \ac{atm} provides \ac{qos} to the network operators and it also allows for the granular control of network paths -- called circuits -- but at a price. The \acp{vc} in an \ac{atm} network have to be tightly managed for the network to function properly, which limits its dynamism and scalability. Together with the Ethernet/\acs{ip} interoperability issues (causing it to move into increasing disuse), \ac{atm} does not make it a viable candidate to implement \acp{dvpn}.

%% SPB %%


Simplicity of setup comes at a cost:
\begin{itemize}
	\item no \ac{te}
	\item limited \ac{ecmp}
	\item no fast failover
\end{itemize}

%% MPLS %%



Because of its extensibility the \ac{mpls} technology and the added protocols and tools, it is commonly used in \acp{cen} as an alternative to legacy \acs{atm} and \acs{sdh} networks. With added features such as \ac{ecmp}, \ac{frr} and explicit routing it has proven to be a technology fit for carriers to transport critical application traffic over large networks

% subsection contemporary_implementations (end)

\subsection{OpenFlow} % (fold)
\label{sub:openflow}

How did it do?

What are the differences?



These features are essential for implementing \acp{vpn} because the traffic flows will need to be separated upon entering the carrier network for MAC scalability.

MAC Scalability only from 1.0 onwards.

1.0 supported almost everywhere, 1.1 and 1.2 are not. 1.3 slowly coming.

The strength of these applications however, is the fact that operators can integrate their \ac{nms} and control plane even more. This allows for a more granular control over their traffic

\ac{mpls} \acp{vpn} in OpenFlow: \cite{mpls-vpn-openflow}

\ac{mpls} control plane in OpenFlow: \cite{mpls-open}


PROS/CONS

% subsection openflow (end)

Also: access layer intelligence.

% section results (end)