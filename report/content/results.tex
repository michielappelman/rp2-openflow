% (c) 2014 Michiel Appelman
\section{Results} % (fold)
\label{sec:results}

The features and limitations of the three discussed technologies in Section~\ref{sec:implementation} are given in Table~\ref{tb:reqs}. In this section we will evaluate them against the requirements set forth in Section~\ref{sec:dvpns}.

\begin{table}[h]
	\centering
	\begin{tabular}{r|lll}
	 & \acs{spb} & \acs{mpls} & OpenFlow / \acs{sdn}\\
	\hline
	Tagging of VPN Traffic & \acs{pbb} & \acs{vpls} & \acs{pbb} / \acs{mpls}\\
	MAC Scalability & yes & yes & yes\\
	Topology Discovery & \acs{isis} & \acs{ospf} & application\\
	Path Provisioning & \acs{spt} & \acs{rsvp} / \acs{ldp} & application\\
	Traffic Engineering & no & \acs{rsvp} & application\\
	\ac{ecmp} & limited & yes & yes, using Groups\\
	\ac{bum} limiting & dependent on \acs{hw} & dependent on \acs{hw} & yes, using Metering\\
	Exchange \acsp{cmac} & no & \ac{evpn} (draft) & application\\
	Traffic Rate Limiting & dependent on \acs{hw} & dependent on \acs{hw} & yes, using Metering\\
	Fast Failover & no & \acs{frr} & yes, using Groups\\
	\acs{oam} & 802.1ag & \acs{lsp} Ping / \acs{bfd} & application\\
	\hline
	Forwarding Decision & \acs{pbb} tags & \acs{mpls} labels & flow entry \\
	\ac{bum} traffic handling & flood & flood & sent to controller\\
	\end{tabular}
	\caption{Feature requirements available in discussed technologies.}
	\label{tb:reqs}
\end{table}


\subsection{\acs{spb}} % (fold)
\label{sub:spb}

Simplicity of setup comes at a cost:
\begin{itemize}
	\item no \ac{te}
	\item limited \ac{ecmp}
	\item no fast failover
\end{itemize}

% subsection spb (end)

\subsection{\acs{mpls}} % (fold)
\label{sub:mpls}

Because of its extensibility the \ac{mpls} technology and the added protocols and tools, it is commonly used in \acp{cen} as an alternative to legacy \acs{atm} and \acs{sdh} networks. With added features such as \ac{ecmp}, \ac{frr} and explicit routing it has proven to be a technology fit for carriers to transport critical application traffic over large networks



% subsection mpls (end)


\subsection{OpenFlow} % (fold)
\label{sub:openflow}

How did it do?

What are the differences?



These features are essential for implementing \acp{vpn} because the traffic flows will need to be separated upon entering the carrier network for MAC scalability.

MAC Scalability only from 1.0 onwards.

1.0 supported almost everywhere, 1.1 and 1.2 are not. 1.3 slowly coming.

The strength of these applications however, is the fact that operators can integrate their \ac{nms} and control plane even more. This allows for a more granular control over their traffic

\ac{mpls} \acp{vpn} in OpenFlow: \cite{mpls-vpn-openflow}

\ac{mpls} control plane in OpenFlow: \cite{mpls-open}


PROS/CONS

% subsection openflow (end)

Also: access layer intelligence.

% section results (end)