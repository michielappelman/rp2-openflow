% (c) 2014 Michiel Appelman
\section{Results} % (fold)
\label{sec:results}

The features and limitations of the three discussed technologies in Section~\ref{sec:implementation} are given in Table~\ref{tb:reqs}. In this section we will evaluate them against the requirements set forth in Section~\ref{sec:dvpns}.

\begin{table}[h]
	\centering
	\begin{tabular}{r|lll}
	 & \acs{spb} & \acs{mpls} & OpenFlow / \acs{sdn}\\
	\hline
	Tagging of VPN Traffic & \acs{pbb} & \acs{vpls} (\acs{mpls}) & \acs{pbb} / \acs{mpls}\\
	MAC Scalability & yes & yes & yes\\
	Topology Discovery & \acs{isis} & \acs{ospf} & application\\
	Path Provisioning & \acs{spt} & \acs{rsvp} / \acs{ldp} & application\\
	Traffic Engineering & limited & \acs{rsvp} & application\\
	\ac{ecmp} & limited & yes & yes, using Groups\\
	\ac{bum} limiting & dependent on \acs{hw} & dependent on \acs{hw} & yes, using Metering\\
	Exchange \acsp{cmac} & no & \ac{evpn} (draft) & application\\
	Ingress Rate Limiting & dependent on \acs{hw} & dependent on \acs{hw} & yes, using Queues or Metering\\
	Fast Failover & no & \acs{frr} & yes, using Groups\\
	\acs{oam} & 802.1ag / Y.1731 & \acs{lsp} Ping / \acs{bfd} & application\\
	\hline
	Forwarding Decision & \acs{pbb} tags & \acs{mpls} labels & flow entry \\
	\ac{bum} traffic handling & flood & flood & sent to controller\\
	\end{tabular}
	\caption{Feature requirements available in discussed technologies.}
	\label{tb:reqs}
\end{table}


\subsection{\acs{spb}} % (fold)
\label{sub:spb}

The \ac{spb} architecture allows for a scalable carrier network supporting thousands, even millions of \acp{dvpn} using the I-SID in the \ac{pbb} frame. From our theoretical implementation, it becomes evident that setting up a \acp{dvpn} requires little configuration. The I-SID only needs to be configured on the head-end \ac{pe} and the \ac{isis} routing protocol distributes this binding to the other corresponding \acp{pe}. Another benefit is the use of Ethernet \ac{oam} standards that are mature and extensive to allow for precise monitoring and troubleshooting.

Unfortunately, the simplicity of the setup comes at a cost. The protocols has:
\begin{itemize}
	\item limited explicit or constraints-based routing, meaning few \ac{te} features,
	\item limited \ac{ecmp} functionality due to the infancy of the standard to support it, and
	\item because, failure recovery depends on \ac{isis} reconvergence, no fast failover.
\end{itemize}

These shortcomings are being worked on by the community, e.g.\ IEEE 802.1Qbp which provides extensive \ac{ecmp} functions. Progress is slow however, and the overall standard will need to mature before it will be suitable for carrier implementations.  

% subsection spb (end)

\subsection{\acs{mpls}} % (fold)
\label{sub:mpls}




% subsection mpls (end)


\subsection{OpenFlow} % (fold)
\label{sub:openflow}

How did it do?

What are the differences?



These features are essential for implementing \acp{vpn} because the traffic flows will need to be separated upon entering the carrier network for MAC scalability.

MAC Scalability only from 1.0 onwards.

1.0 supported almost everywhere, 1.1 and 1.2 are not. 1.3 slowly coming.

The strength of these applications however, is the fact that operators can integrate their \ac{nms} and control plane even more. This allows for a more granular control over their traffic

\ac{mpls} \acp{vpn} in OpenFlow: \cite{mpls-vpn-openflow}

\ac{mpls} control plane in OpenFlow: \cite{mpls-open}

Lacks liveness monitoring

PROS/CONS

% subsection openflow (end)

Also: access layer intelligence.

% section results (end)