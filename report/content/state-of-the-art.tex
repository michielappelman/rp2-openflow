% (c) 2014 Michiel Appelman
\section{State of the Art} % (fold)
\label{sec:state_of_the_art}

The issue of creating dynamic \acp{vpn} within \ac{sp} networks apparently comes from the inability to do so using technologies available to operators today. To get an understanding of how we got to this point and where the limitations or obstacles lie, an overview of the state of the art is required.

To be qualified to provide network operators with dynamic \acp{vpn} each technology will need to be able to provide the following features:

\begin{enumerate}
	\item scalable up to thousands of (dynamic) Layer 2 \acp{vpn} and client \acsp{mac},
	\item fast failover times (<50ms) to provide continuity to critical applications,
	\item efficient use of, and control over all network resources,
	\item provide \acl{qos} features to differentiate between classes of applications, and
	\item an automated way to install \acp{vpn} in the network.
\end{enumerate}

\subsection{\acs{sdh}/\acs{sonet}} % (fold)
\label{sub:sdh_sonet}
The foundation of \acp{wan} today is still \ac{sdh} and its North-American counterpart \ac{sonet}. It operates at Layer 1 and is concerned with putting various data streams on fiber links. \ac{sdh} networks find their origin in the telecommunications world where it was developed to transfer real-time cell switched voice data. It quickly evolved to include \ac{atm} and later also Ethernet. Due to its maturity the protocol is considered stable but is also very static in its manageability. Developments are ongoing in the \ac{gmpls} field to bring dynamic configuration to \ac{sdh} devices, but they are in their infancy.

% subsection sdh_sonet (end)

\subsection{\acs{atm}} % (fold)
\label{sub:atm}
\ac{atm} is a legacy protocol that has been used by operators to carry traffic over the internet backbone since the 1990s. Where as Ethernet and \ac{ip} are developed as packet-routing connection-less protocols, \ac{atm} is a cell-switched connection-oriented protocol. This poses a number of problems when trying to transport \ac{ip} over \ac{atm}. First, the variable length of the packets don't map efficiently to the fixed size cells of \ac{atm}. Drops of a single cell would cause the entire frame to become unusable. Then there is the added overhead of encapsulating \ac{ip} over \ac{atm}, which causes inefficient use of the network resources when compared to running an all \ac{ip} network. Finally, the \ac{qos} features of \ac{atm} are left unused. These problems are some of the reasons that operators have moved away from \ac{atm} based backbones, to all \ac{ip} ones. 

As mentioned, \ac{atm} does provide \ac{qos} to the network operators and it also allows for the granular control of network paths -- called circuits -- but at a price. The \acp{vc} in an \ac{atm} network have to be tightly managed for the network to function properly. This limits its dynamism and scalability, which (also considering the aforementioned objections) does not make it a viable candidate to implement dynamic \acp{vpn}.

% subsection atm (end)

\subsection{\acs{mpls}} % (fold)
\label{sub:mpls}
\ac{mpls} is known for its scalability and extensibility. Over the past decade additions have been made to the original specification to overcome a plethora of issues within carrier networks. This initially started with trying to implement fast forwarding in legacy switches using labels (or tags) at the start of the frame \cite{tag-switching}. When this issue became surmountable using new hardware, \ac{mpls} had already proven to be capable of transporting a wide arrange of protocols on the carrier backbone network, all the while also providing scalability, \ac{te} and \ac{qos} features to the operators.

\ac{mpls} itself is more a way of forwarding frames through the network, without facilitating any topology discovery, route determination, resource management, etc. These functions are left to a stack of other protocols. To discover the topology, \ac{mpls} relies on an \ac{igp}. The distribution of labels is done using \ac{ldp} and/or \ac{rsvp}, of which the latter also provides granular \ac{te} and \ac{qos} functionalities.

\acp{vpn} are also provided by additional protocols. Layer 3 \acp{vpn} make use of \ac{bgp} to distribute client prefixes to the edges of the carrier network. The core is only concerned with the forwarding of labels and has now knowledge of these \acs{ip} prefixes. Layer 2 \acp{vpn} make use of \ac{ldp} and \ac{vpls}, a service which encapsulates the entire Ethernet frame and pushes a label to it to map it to a certain separated network. Again, the core is only concerned with the labels and only the edges need to know the clients \acs{mac} addresses. 

Because of its extensibility the \ac{mpls} technology and the added protocols and tools, it is commonly used in \acp{cen} as an alternative to legacy \acs{atm} and \acs{sdh} networks. With added features such as \ac{ecmp}, \ac{frr} and explicit routing it has proven to be a technology fit for carriers to transport critical applicition traffic over large networks. \ac{mpls} thus meets the requirements set forth, which might be the reason for its widespread use presently. However, given the scale and criticality of the networks and the stack of various protocols to implement it, management of \ac{mpls} networks has remained fairly static and mostly a manual task.

% subsection mpls (end)

\subsection{\acs{spb}} % (fold)
\label{sub:spb}
\ac{spb} is an evolution of the original \acs{ieee} 802.1Q \ac{vlan} standard. \ac{vlan} tags have been in use in the networking world for a long time and provide decent separation in campus networks. However, when \ac{vlan}-tagging was done at the customer network, the carrier couldn't separate the traffic from different customers anymore. This resulted in 802.1Qad or Q-in-Q which added an S-\ac{vlan} tag to separate the client \acp{vlan} from the \ac{sp} \acp{vlan} in the backbone. This was usable for the Metro Ethernet networks for awhile but when \acp{sp} started providing this services to more and more customers, their backbone switches could not keep up with the clients \ac{mac} addresses.

To solve this scalability problem \ac{pbb} (802.1Qay or \ac{mac}-in-\ac{mac}) was introduced. It encapsulates the whole Ethernet frame on the edge of the carrier network and forwards the frame based on the Backbone-\ac{mac}, Backbone-\ac{vlan} and the I-SID. The I-SID is a Service Instance Identifier, which with 24 bits is able to supply the carrier with 16 million separate networks. The downside of \ac{pbb} remained one that is common to all Layer 2 forwarding protocols: the possibility of loops. Preventing them requires \ac{stp} which will disable links to get a loop-free network. Disadvantages of \ac{stp} include the relatively long convergence time and inefficient use of resources due to the disabled links. This final problem was solved by using \acs{isis} as a routing protocol to distributed the topology and creating \acp{spt} originating from each edge device. This is called \ac{spb} or 802.1aq.

\ac{spb} benefits from the maturity of the Ethernet protocol by reusing protocols for \ac{oam} and \ac{pm} and the fact that only the edges of the network need to be \ac{spb} capable -- the core switches just need to be able to forward 802.1Qad frames. Manageability of \ac{spb} \acp{vpn} is simplified to mapping a certain customer port or \ac{vlan} to an I-SID. Limitations of \ac{spb} include a very coarse way to apply \ac{te} policies or \ac{ecmp} load sharing. Although drafts are in the works to resolve this limitation, this has yet to be standardized and implemented. Finally, the protocol at this point lacks fast failovers, and as such will not be a viable option to implement dynamic \acp{vpn}.


% subsection spb (end)

\subsection{\acs{trill}} % (fold)
\label{sub:trill}
There has been discussion going on between \ac{spb} and \ac{trill} supporters as to which is the `better' protocol. Indeed, both use \ac{isis} as an \ac{igp} and both try to solve the same problem to make Ethernet networks scalable to the desired scale of networks in use today, but they differ in their premise and implementation. \ac{trill} adds a completely new header on top of  the Ethernet frame with a source and destination RBridge allowing the so-called RBridges to route -- Layer 3 -- the frame to its destination over the \ac{trill} backbone network.

Without this paper turning into another \ac{trill} versus \ac{spb} comparison, it is important to point out various differences in the two technologies with regards to our proposed use-case. First, the hop-by-hop decision making by \ac{trill} has the benefit of being able to distribute traffic in more efficient way over a dense network. \ac{spb} on the other hand makes its forwarding decision at the head-end, similar to \ac{mpls} forwarding. This means that careful consideration needs to take place to choose either more efficient \ac{ecmp} forwarding or granular control over forwarding paths. 

Additionally, because \ac{trill} is a completely new protocol, the \ac{ietf} is also still working on adding some essential features that are still in draft. These include \ac{qos}, \ac{oam}, \ac{pm} and \ac{te}. And finally there is another limitation for \acp{cen}, which is the lack of network separation. \ac{trill} only supports the 4096 \ac{vlan} tags present in the 802.1Q frame to separate customer networks. At this point, it will fail to meet the requirements to provide dynamic \acp{vpn}.

% subsection trill (end)


\subsection{OpenFlow} % (fold)
\label{sub:openflow}
Section~\ref{sec:introduction} gave a short introduction into what \ac{sdn} and OpenFlow entail and what it promises in terms of cost savings and agility. \acl{sdn} is the general principle of designing flexible networks using open interfaces towards the hardware. The complete architecture is being standardized by the networking industry within the OpenDaylight project \cite{opendaylight}. Looking at OpenFlow itself from a more technical point of view, it boils down to a protocol used to program the forwarding plane in networking devices from a centralized server called the `controller'. 

The momentum that \ac{sdn} is getting might be explained by a general need for change in the networking industry. Operators primarily want to get more control over their networks, something which using the current stack of protocols is relatively complicated to get. The original \acs{osi} reference model \cite{zimmermann} touches on the ``Management Aspects'' of each layer in the model, a way for management entities in the highest layer to control the behavior of lower layers. Unfortunately in the swift evolution of \ac{tcp}/\ac{ip}, these management interfaces are often limited or absent all together. 

The OpenFlow controller provides operators with an alternative to these interfaces, namely a programmable forwarding plane. This gives them \ac{te} and \ac{qos} control by using custom applications to direct traffic through the network. 

The OpenFlow specification has been through a few revision since it was first proposed. In the first version tagging of traffic was only support using a single \acs{vlan} tag. Version 1.1 added matches for \ac{mpls} and Q-in-Q tags, and version 1.3 could also match \ac{pbb} tags. These features are essential for implementing \acp{vpn} because the traffic flows will need to be separated upon entering the carrier network. Also, with scalability in mind we do not want to learn the client \acsp{mac} in the core of the network.

Also not included in version 1.0 were mechanisms to allow for fast failover and \ac{ecmp}. These were added in version 1.1 by introducing logical groups of ports to which a flow entry can forward the frame. However, to support fast failover a \textsl{liveness monitoring} technique will need to be implemented supported by the switch. This could simply be the physical link state or more intricate tools, e.g.\ \ac{bfd}.

% subsection openflow (end)

% section state_of_the_art (end