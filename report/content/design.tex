% (c) 2014 Michiel Appelman
\section{Design} % (fold)
\label{sec:design}

\subsection{Requirements} % (fold)
\label{sub:requirements}
\begin{enumerate}
	\item What will the system need to look like?
	\item What are its functional requirements?
	\item What is the input?
	\item What is the expected output?
	\item How can it be extended in a later stage?
\end{enumerate}

To further shape the actual design of the system, we will specify what this \ac{dvpn} service will need to look like. 

First, we will need to define a certain input. Our design accepts a set of physical ports with an optional C-\ac{vlan} corresponding to it. This set represents the group of users which will be placed in a single \ac{vpn} and thereby connected to each other. It also requires the values for minimum and maximum bandwidth used by \ac{dvpn} over the network. High bandwidth \acp{dvpn} may need load-sharing over multiple physical 1GbE links for example.

Second, the output will be defined as the \ac{vpn} created throughout the network and Layer 2 connectivity between the chosen endpoints from the input.

Third, the usage of the network resources should be monitored during the lifetime of the \ac{dvpn}. If certain paths are nearing their capacity, \acp{dvpn} should be able to be moved to paths where more resources are available.

Finally, the tearing down of the \ac{dvpn} will also need to be arranged so as to free up resources for new \acp{dvpn}.

A design for the complete process will be given in the following sections, first in a situation where traditional network management and \ac{mpls} are used. And after that using the \ac{sdn} approach and its way of managing network devices.

% subsection requirements (end)

\subsection{\acs{mpls}} % (fold)
\label{sub:contemporary}
To start:
\begin{enumerate}
	\item Input goes where?
	\item Flow of information.
	\item Amount and type of output.
\end{enumerate}

As has been discussed in Section~\ref{sub:mpls} a typical \ac{mpls} network consists of a stack of routing protocols. This means that to provision the \ac{dvpn} several protocols will be affected. 

First, determine best path and configure \ac{rsvp} to make paths between \acp{pe}. Needs \ac{te} input. 

Second, \ac{vpls} to add the ports to the \ac{vpls} instance and define the paths to use.

LDP monitored for adjacency between \acp{pe}.

In the access layer, \ac{mpls} has to be supported. Otherwise no mapping to customer \ac{vlan}, only per port. Other option: Q-in-Q but mapping of C AND S-TAG to VPLS instance needs to be supported by \ac{pe}.

% subsection contemporary (end)

\subsection{OpenFlow} % (fold)
\label{sub:openflow}
To start:
\begin{enumerate}
	\item Input goes where?
	\item Flow of information.
	\item Amount and type of output.
\end{enumerate}



Need OpenFlow support on access-layer, or Q-in-Q to map to VPLS instance at \ac{pe}, OR match on \ac{mac}.



% subsection openflow (end)



% section design (end)
