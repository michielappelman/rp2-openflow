\section{Dynamic VPNs} % (fold)
\label{sec:dvpns}

The 


To further shape the actual design of the system, we will specify what this \ac{dvpn} service will need to look like. 

First, we will need to define a certain input. Our design accepts a set of physical ports with an optional C-\ac{vlan} corresponding to it. This set represents the group of users which will be placed in a single \ac{vpn} and thereby connected to each other. It also requires the values for minimum and maximum bandwidth used by \ac{dvpn} over the network. High bandwidth \acp{dvpn} may need load-sharing over multiple physical 1GbE links for example.

Second, the output will be defined as the \ac{vpn} created throughout the network and Layer 2 connectivity between the chosen endpoints from the input.

Third, the usage of the network resources should be monitored during the lifetime of the \ac{dvpn}. If certain links are nearing their capacity, \acp{dvpn} should be able to move paths to links where more resources are available.

Finally, the tearing down of the \ac{dvpn} will also need to be arranged so as to free up resources for new \acp{dvpn}.


wat zijn het? waarom?

wat moet je weten uit het netwerk?

wat voor input heb je nodig?


requirements:
tagging - why?
ecmp - why?
oam - why?
te - why? 
rate limiting - why? (also on BUM traffic)
arp proxy - why?

hoe doe je routing?

what information from the network?
links
link utilization


nms information bases:
mac/ip address management?
dvpns and ports and paths
paths and links


paden:
hoe bepaal je ze?

qos:
hoe doe je rate limiting?



The issue of creating \acp{dvpn} within \ac{sp} networks apparently comes from the inability to do so using technologies available to operators today. 

To be qualified to provide network operators with \acp{dvpn} each technology will need to be able to provide the following features:

\begin{enumerate}
	\item scalable up to thousands of (dynamic) Layer 2 \acp{vpn} and client \acsp{mac},
	\item fast failover times (<50ms) to provide continuity to critical applications,
	\item efficient use of, and control over all network resources,
	\item provide \acl{qos} features to differentiate between classes of applications, and
	\item an automated way to install \acp{vpn} in the network.
\end{enumerate}

% section dynamic_vpns (end)