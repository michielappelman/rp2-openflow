% (c) 2014 Michiel Appelman
\vspace*{\fill}
{\section*{Abstract}\addcontentsline{toc}{section}{Abstract}
\label{sec:summary}

%motivatie
%Todays network operators make extensive use of \aclp{nms} to manage and control their networks, which are growing in size and complexity. The amount of \acsp{vpn} which are used to provide client connectivity and to manage network resources, is growing as well. 
The increasing complexity of networks and \acsp{nms} is starting to affect operators, who are seeing a growing demand for \aclp{dvpn}. \acsp{dvpn} are application-specific \acsp{vpn} which can be altered multiple time over their potentially short lifetime, requiring a certain degree of flexibility and agility from the network and its support systems.

To implement \acsp{dvpn} in the network, operators need to solve the complexity of \acsp{nms} and allow for granular control over network resources. A possible candidate to provide this solution is the \acs{sdn} architecture and the OpenFlow specification. However, it is unclear if this solution will actually provide any benefit over the use of state of the art technologies.
%oplossing

This research compares the differences between implementing a \acs{dvpn} service using the contemporary \acs{mpls} stack and implementing it using OpenFlow. 
We found that the \acs{mpls} implementation can provide the \acs{vpn} service but due to its large protocol stack and lack of a defined management interface, will prove to be unsuitable when implementing \acsp{dvpn}. 

On the other hand, the \acs{sdn} architecture can solve complexity and provide manageability by providing network abstractions to applications which can be developed by the operators themselves. However, until the northbound and east/westbound interfaces are defined, portability and flexibility is still limited. 

Additionally, this research shows that OpenFlow is missing monitoring in its forwarding plane allowing for individual components to make independent choices to provide fast failover times. This limitation means that the networking devices will need support from the controller to detect faults in the path, yielding recovery times above operator requirements.

% section summary (end)
}
\vspace*{\fill}