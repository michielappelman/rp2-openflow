% (c) 2014 Michiel Appelman
\section{Introduction} % (fold)
\label{sec:introduction}
Network operators today use \acp{nms} to get control over their devices and services that they deploy. These systems have been customized to their needs and in general perform their functionalities adequately. However, operators run into obstacles when trying to expand their business portfolio by adding new services. This will require 
\begin{inparaenum}[\itshape a\upshape)]
	\item new \ac{api} calls to be implemented between their B/OSS and \ac{nms}, 
	\item their \ac{nms} to be able to cope with potentially new protocols, and
	\item added expertise by engineers to define the requirements and restrictions of these protocols. 
\end{inparaenum}
When these obstacles are eventually overcome the setup that will result from this implementation will be relatively static, since any change to it will require the whole process to be repeated.

Until recently this limitation didn't distress operators as their networks were in fact primarily static. But with increasing demand for services requiring for example mobility and short-term virtual networks, these limitations start to become a tangible problem for operators. By solving the complexity of implementing new services or features for them, they will be able shorten their time to market, save on networking expertise and be more adaptive to changes in these services. 

A potential candidate to solve this complexity is OpenFlow \cite{openflow} and \ac{sdn}. \ac{sdn} is a relatively new architecture to allow for the programmability of networks. The architecture has recently been standardized in the OpenDaylight project \cite{opendaylight} which also includes OpenFlow, a lower level and increasingly supported \ac{api} towards networking devices. Implementing the \ac{sdn} architecture promises
\begin{inparaenum}[\itshape a\upshape)]
	\item CAPEX savings due to hardware being more generic and flexible,
	\item OPEX savings because of the integration of \acp{nms} and the control interface of the devices, thereby increasing automation, and
	\item increased network agility by using the open interfaces to program network devices directly \cite{packet-circuit}.
	\end{inparaenum} 

	\subsection{Research Question} % (fold)
	\label{sub:research_question}
	It is unclear however if a real-world OpenFlow and \ac{sdn} implementation will actually provide any simplicity, additional flexibility or cost savings when compared to contemporary technologies \cite{programmability-answer}. Indeed, the technologies in use today have served operators well up until this point and their practicality has been proven over the past years. 
	This research will seek to identify where exactly operators can benefit from implementing this use-case using \ac{sdn} compared to the architecture in use today. 
	
	This research offers that -- given the use case as defined in Section~\ref{sub:scope} -- OpenFlow will reduce the complexity in the architecture of the management systems and the network as a whole. 
	To prove this, we will need to answer the question: \textsl{``How much can operators benefit from using OpenFlow when implementing \aclp{dvpn}?''} 
	

	% subsection research_question (end)

	\subsection{Scope} % (fold)
	\label{sub:scope}
	\acp{dvpn} are private networks over which end-users can communicate, deployed by their common \ac{sp}. They differ from normal \acp{vpn} in the sense that they are relatively short-lived. Using \acp{dvpn}, \acp{sp} can react more swiftly to customer requests to configure, adjust or tear down their \acp{vpn}.
	This research will prove if such a service can be implemented using contemporary technologies. And, if so, what such a network will look like with regards to the protocols needed.
	
	 More importantly, we will compare the characteristics of implementing such an environment using both available technologies and an \ac{sdn} solution. 
	%This will include research to determine if the proposed \ac{sdn} and OpenFlow architecture will actually be able to support the creation of \acp{dvpn}.
	The focus will primarily be on deploying \acp{ppvpn} at Layer 2 of the \acs{osi}-model between end-users. We haven chosen to do so because these Ethernet \acp{vpn} are characterized by their transparency to the end-user, who will be placed in a single broadcast domain with its peers and can thus communicate directly without configuring any sort of routing.
	
	Previous research in \cite{net-prog-vpn} has proposed a very specific implementation for programmable networks to deploy on-demand \acp{vpn} but it predates the OpenFlow specification, and also omits a comparison with how this would look using contemporary technologies.

	% subsection scope (end)

	\subsection{Approach} % (fold)
	\label{sub:approach}
	In the Section~\ref{sec:dvpns} we will define the conceptual design of \acp{dvpn}. This will result in a list of required features for the technologies to provide such a service. Section~\ref{sec:implementation} will list the technologies available and will additionally determine their usability for implementing \acp{dvpn} when taking into account the requirements set forth in Section~\ref{sec:dvpns}. In Section~\ref{sec:results} we will distill the advantages and limitations of the different implementations and substantiate how the compare to each other. Finally, Section~\ref{sec:conclusion} summarizes the results and provides a discussion and future work on this subject.

	% subsection approach (end)

% section introduction (end)